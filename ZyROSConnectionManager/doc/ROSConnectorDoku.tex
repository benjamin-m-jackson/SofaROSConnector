\documentclass[a4paper, 10pt]{article}

%textheight in cm: 19.32755cm
%textwidth in cm: 12.12364cm

\usepackage{fullpage}
%\usepackage[margin=100pt]{geometry}

\usepackage[T1]{fontenc}
\usepackage{underscore}

\usepackage{graphicx}
\usepackage{listings}


\pagestyle{empty}
\begin{document}
\lstset{language=[Visual]C++}
\section*{TruPhysics ROSConnector documentation (\today)}
    \subsection*{Subscribing to topics}
    \begin{itemize}
        \item Needs the TruRosConnectionManager-plugin (TruRosConnectionManager.dll) and additionally TruRosConnector.dll

        \item Add a TruRosConnectionManager-Element to the scene graph. This element has the following two parameters:
            \begin{itemize}
                \item rosMasterURI: A ROS master URI
                \item rosTopics: (optional) A list of subscribed ROS-topics. A list entry has the following format: <topic name>:::<message type>. List entries are separated by ';'
            \end{itemize}
        \item Example:\\
            <TruRosConnectionManager name="truPhysicsRos" \\
            rosMasterURI="http://10.1.1.3:11311" \\
            rosTopics="/joint_states:::sensor_msgs::JointState;/gripper_state:::std_msgs::Bool"/>
    \end{itemize}
            
    Currently, the TruRosConnectionManager supports subscribing to messages of type 'sensor_msgs::JointState' and 'std_msgs::Bool'. To implement subscribing to a new message type, a few lines of code that instantiate the subscriber class when needed must be added to the function TruRosConnectionManager::bwdInit(). For std_msgs::Bool messages this is:
\begin{lstlisting}[frame=single]
if (messageType.compare("std_msgs::Bool") == 0)
{
  supported = true;

  boost::shared_ptr< TruRosConnectorTopicSubscriber<std_msgs::Bool> > tmp(
    new TruRosConnectorTopicSubscriber<std_msgs::Bool>(
      m_ros_connector->getROSNode(),
      rosTopic
    )
  );
  //add a message queue length (default: 50) and 'true' to the constructor
  //parameters of the TruRosConnectorTopicSubscriber, to get a subscriber
  //that writes every incoming message to the command line (for testing)

  m_ros_connector->addTopicListener(
    boost::dynamic_pointer_cast<TruRosListener>(tmp)
  );
  topicListeners.push_back(
    boost::dynamic_pointer_cast<TruRosListener>(tmp)
  );
}
\end{lstlisting}        
For other message types, this can be adapted easily.
    
\subsubsection*{Processing messages to subscribed topics}
When a message is published to a subscribed topic, the subscriber class emits a boost::signal2. Connecting to this signal is currently work in progress. The file connectToSubscriberSignalExample.txt contains a code example of how it works currently.

\clearpage
\subsection*{Publishing topics}
\begin{itemize}
    \item Needs the TruRosConnectionManager-plugin (TruRosConnectionManager.dll) and also TruRosConnector.dll and TruRosPublishingHandler.dll
    
    \item Add a TruRosConnectionManager-Element to the scene graph (same as for subscribing, see above). Note that published rostopics do not need to be given as parameters of the TruRosConnectionManager-Element.

    \item The project in which messages are to be published needs to link against TruRosConnector and TruRosPublishingHandler and include the directories with their headers.
    
    \item The class that is supposed to publish messages needs to include 'TruPhysicsROSConnector.h', 'TruRosPublishingHandler.h'. The publisher then needs to be initialized once, before messages can be published.
    
    \item to initialize the publisher:
    \begin{itemize}
        \item first instantiate an object of class TruRosPublishingHandler and call the 'setROSConnectionManagerByContext(BaseContext*)' method with a scene-graph context as an argument.
        
        The method looks for a TruRosConnectionManager in the scene graph and returns true if successful.
        
        (CAREFUL! When this is called, the TruRosConnectionManager in the scene graph must already be instantiated, so a constructor or an init() method are bad places to call setROSConnectionManagerByContext - I usually use a bwdInit() method)
        \item instantiate the TruRosConnectorTopicPublisher object, using the message-type as template-argument. The constructor-arguments are the 'getROSNodeHandle()' method of the TruRosPublishingHandler and a string with the name of the topic to publish.
        \item call the 'registerPublisher(ROSConnector::TruRosPublisher*)' method of the TruRosPublishingHandler, using a pointer to the publisher object as argument
    \end{itemize}
        
    \item to publish a message: 
    \begin{itemize}
        \item create an object of the message type that is supposed to be published and fill it with data
        \item call the 'publishMessage' method of the publisher
    \end{itemize}

\end{itemize}
\clearpage
\noindent Example, initialization: 
\begin{lstlisting}[frame=single]
TruPhysics::ROSPublishing::TruRosPublishingHandler publishingHandler;
TruPhysics::ROSConnector::TruRosConnectorTopicPublisher<
  std_msgs::Float32
>* timePublisher;
    
connectionManagerFound =
  publishingHandler.setROSConnectionManagerByContext(
    getContext()
  );

if (connectionManagerFound)
{
  timePublisher =
    new TruPhysics::ROSConnector::TruRosConnectorTopicPublisher<
      std_msgs::Float32
    >
    (
      publishingHandler.getROSNodeHandle(),
      "sofaSimTime"
    );
  publishingHandler.registerPublisher(timePublisher);
}
\end{lstlisting}
Example, publish message:
\begin{lstlisting}[frame=single]
if (connectionManagerFound)
{
  // publish simulation time
  std_msgs::Float32 msgFloat32;
  msgFloat32.data = getContext()->getTime();
  if (timePublisher)
  {
    timePublisher->publishMessage(msgFloat32);
  }
}
\end{lstlisting}    

\end{document}
